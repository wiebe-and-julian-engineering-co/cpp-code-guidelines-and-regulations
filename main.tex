\documentclass[12pt]{article}

\usepackage[style=ieee]{biblatex}
\addbibresource{main.bib}

\usepackage[T1]{fontenc}
\usepackage{lmodern}
\renewcommand*\familydefault{\sfdefault}

\usepackage{listings}

\usepackage{hyperref}
\hypersetup{
    colorlinks,
    citecolor=black,
    filecolor=black,
    linkcolor=black,
    urlcolor=black
}

\usepackage[margin=1in]{geometry}
\usepackage{fancyhdr}

\pagestyle{fancy}
\lhead{}
\chead{}
\rhead{}
\lfoot{\footnotesize Draft}
\cfoot{}
\rfoot{\thepage}
\renewcommand{\headrulewidth}{0pt}
\renewcommand{\footrulewidth}{0pt}

\newcommand{\cppuntil}[1]{\footnotesize\textit{(until C++#1)}\normalsize}
\newcommand{\cppsince}[1]{\footnotesize\textit{(since C++#1)}\normalsize}
\newcommand{\requirementpriority}[1]{\emph{Required-Level #1}}


\newcounter{coderulecounter}

\newenvironment{coderule}{%
\begin{samepage}%
\refstepcounter{coderulecounter}%
\newcommand{\rulename}[2]{%
\addcontentsline{toc}{subsubsection}{Rule \thecoderulecounter: ##1 - \emph{##2}}%
\subsection*{Rule \thecoderulecounter: ##1 - \emph{##2}}%
}%
\newcommand{\rationale}{\subsubsection*{Rationale}}%
\newcommand{\alternatives}{\subsubsection*{Alternatives}}%
}{%
\end{samepage}%
\bigbreak\bigbreak%
}

\title{
      C++ Code Guidelines and Regulations \\
      \begin{large}
         Safety-Critical C++ for Embedded Systems
      \end{large}
}
\date{\today}
\author{Wiebe and Julian Engineering Cooperation}

\begin{document}
\lstset{language=C++}

\pagenumbering{gobble} % No numbering on title page
\maketitle

\newpage
\pagenumbering{Roman} % Required to be roman until table of contents
\tableofcontents
\newpage
\pagenumbering{arabic} % Required to be arabic after table of contents

\newpage
\section{Introduction}
C++ is a general-purpose programming language. It has a wide vocabulary of
features, some of which are inspired by other language and some inspired other
languages. C++ is commonly called a superset of C, which in many respects
is true.

Because of the nature of being a superset of C, C++ has many
close-to-the-hardware programming capabilities. These low-level capabilities
must be handled with discipline and sanity. This document is used as a guide
and reference for C++ programmers in \textit{Wiebe and Julian Engineering Co.}
to achieve sane and safe C++ programs.

Many C++ programming aspects are evaluated in this document, included but not
limited by: readability, maintainability, portability, efficiency and security. \\
C++ code using the hwstl guidelines is also referred to as \textit{hwstl C++}.
Said C++ code is meant to be conform to all rules that are per definition
\requirementpriority{0} \\
The following topics will be covered in the C++ Code Guidelines and regulations
for Safety-Critical C++ for embedded sytems:
\begin{enumerate}
   \item Using the C++ standard in embedded systems,
   \item writing sane and safe C++ code, and
   \item following design goals for well written production code.
\end{enumerate}
\subsection{Requirement Levels}
Each formulation of a rule has an associated requirement priority, the
following priorities are formulated:
\begin{enumerate}
   \item \requirementpriority{0}: every rule with this priority is mandated to
be followed by all \textit{hwstl C++}.
   \item \requirementpriority{1}: all \textit{hwstl C++} made by \textit{Wiebe
and Julian Engineering Co.} is mandated to follow this rule. Exceptions can
only be made after inquiring a lead engineer.
   \item \requirementpriority{2}: it is advised to follow rules with this
priority level. Cases where it might be better to not follow the rule are not
required to be rapported.
\end{enumerate}

\pagebreak
\subsection{Guideline Formulation Rationale}
\subsubsection{Functional Safety}
Functional safety is evaluated using Peter Sommerlad's model of safety and
sanety.\cite{SANE_AND_SAFE_CPP_CLASSES} It is an informal representation of
segmenting structs and functions into their respective categories in terms of
memory safety and sanity.

<insert-chart>

The guidelines will try to maintain sanity and safety throughout the guides as
much as possible. Rules are classified by the following categories:

\begin{enumerate}
   \item \emph{Sane:} it is a sensible thing to use, it is clear what the
   desired behavior is;
   \item \emph{Safe:} it is a safe thing to use, it is highly unlikely to cause
   system instability;
   \item \emph{High-discipline:} when strict rules are not abided when
   programming, system instability is likely;
   \item \emph{Ill-advised:} it is ill-advised to write, it is most definitely
   unclear what the desired behavior is;
   \item \emph{Dangerous:} little programming errors can cause immediate system
   instability
\end{enumerate}

\pagebreak
\section{C++ Code Guidelines and Regulations}
\subsection{Reading Guide}
This section describes how the guides and regulations should be read and
interpreted. There are a couple of keywords that denote the urgency of
following a requirement. Such as:
\bigbreak
\textbf{Test:} Hello world
\subsubsection{C++ Standards}
This guide takes modern C++ as of C++17 into account for coding practices.
However, a platform may not have any compilers available with support for the
latest C++ standard. Some practices may be banned in C++17 but are advised to
use in C++14 or earlier standards, rules describing practices that are not
supported by different versions are annotated with \cppuntil{14} or
\cppsince{17} targs. These tags can contain any C++ standard and pre-C++11
annotations may be omitted due to assumed availability of newer standards on a
given platform.

\pagebreak
\section{Hwstl Rules}
This section describes all the rules that are required to be followed to write
hwstl conform C++ code. Different levels of conformity are available, see
INSERT\_REFERENCE\_TO\_CONFORMITY\_LEVELS

\subsection{Variable naming}
\begin{coderule}
\rulename{Template parameter name prefix}{Sane and safe}
Template parameter names must be prefixed with \lstinline{t_} for singular types
and values or \lstinline{vt_} for variadic template parameters.

Example:

\begin{lstlisting}
template <class t_value>
struct compilation_constant {
   using value = t_value;
};

template <class... vt_values>
struct compilation_constant_sequence {
   void apply() {
      std::array a = { vt_values... };

      // use a in this context
   }
};
\end{lstlisting}

\rationale
It is often a convention to write templates with only uppercase characters. But
this does not make the distinction between preprocessor macros. Also types,
values and variable length parameters have distinct behavior. By writing
\lstinline{t_} it is clearly expressed that the parameter in question is a
template constant.

Because of often unorthodox usages of templates, it is required to be more
expressive where variadic parameters, prefixed with \lstinline{vt_} are used.
Since variadic template parameters can be expanded in various different manners,
using, for example, fold expressions \cppsince{17}, it may not be obvious what
argument is expanded.

\end{coderule}

\subsection{Pointer Semantics}

\begin{coderule}
\rulename{Smart Pointer Input Parameters}{High-Discipline}
Usage of rvalue reference input parameters is considered \emph{High-discipline}.
An example snippet in question goes as following:
\begin{lstlisting}
/**
 * Tries to claim moved, may fail to claim which will return
 * a nullptr
 */
template <class T>
hwstl::observer_ptr<T> Claim(std::unique_ptr<T>&& moved) {
   if (can_claim) {
      my_internal_ptr = moved;
      // Tell the caller that Claim was successful. An observer
      // is returned.
      return hwstl::make_observer<T>(MyInternalPtr.get());
   } else {
      // Tell the caller that Claim has failed, the unique_ptr
      // which was meant to be claimed is not claimed. But not
      // destructed either.
      return nullptr;
   }
}
\end{lstlisting}
It must be noted that the above code uses an rvalue reference to claim
ownership, so it does not destruct the claimed ptr. Which clearly differentiates
from a function with the signature \begin{lstlisting}
hwstl::observer_ptr<T> Claim(std::unique_ptr<T> moved)
\end{lstlisting} which destructs moved when a failure occured.

\rationale
None

\alternatives
The usage of \lstinline{unique_ptr} and \lstinline{shared_ptr} is advised for
the appropriate use cases. When using hwstl, \lstinline{hwstl::shared_ptr} is
available, which is advised to use for small embedded devices.
\end{coderule}

\begin{coderule}
\rulename{Owning Raw Pointers}{Dangerous}
Usage of owning raw pointers is considered \emph{Dangerous}

\rationale
Owning raw pointers cuickly leave uncleared memory resources what during
extended periods of operation cause system instability.

\alternatives
The usage of \lstinline{unique_ptr} and \lstinline{shared_ptr} is advised for
the appropriate use cases. When using hwstl, \lstinline{hwstl::shared_ptr} is
available, which is advised to use for small embedded devices.
\end{coderule}

\begin{coderule}
\rulename{Non-Owning Raw Pointers}{Ill-advised}
The usage of non-owning raw pointers is considered \emph{Ill-advised}

\rationale
Non-owning raw pointers are unclear when used in isolated context.

\alternatives
\begin{enumerate}
\item When a pointer does not manage any resources, but is nullable
\lstinline{hwstl::observer_ptr} is required to use.
\item When a pointer is not nullable a reference is required to use.
\end{enumerate}
\end{coderule}

\pagebreak
\section{Bitfields}
\begin{coderule}
\rulename{Signed Bit Fields}{Dangerous}
Signed bit fields are banned, unless they are required to represent a negative
value and contain $> 1$ bits.

\rationale
\begin{figure}[h]
\begin{lstlisting}
struct BitFlags {
  BitFlags(int x, int y, int z) : x(x), y(y), z(z) { }
  int x : 1;
  int y : 1;
  int z : 1;
};

int main() {
  auto flags = BitFlags(1, 0, 1);

  // The following statement prints -1, due to signedness, 
  // while it may not be expected. This leads to hard to spot
  // errors. 
  std::cout << flags.x << std::endl;
}
\end{lstlisting}
\caption{Dangerous Signed Bitfields}
\label{fig:DANGEROUS_SIGNED_BITFIEDS}
\end{figure}

In figure \ref{fig:DANGEROUS_SIGNED_BITFIEDS} it is expressed how signed single
bit bitfields can cause logic errors when programming. This is unexpected
because constructing a bitfield \lstinline{int x : 1;} with \lstinline{x(1)},
the logical 1 is implicitly casted to -1. Due to this unexpected behavior signed
bitfields are in general ill-advised and in the case of 1 bit signed bitfields
it is simply dangerous.

\alternatives
\begin{enumerate}
\item Using \lstinline{unsigned}, \lstinline{unsigned int} or any unsigned type in
favor of signed types is preferred.
\item In the case of 1-bit sized bit-fields, a boolean is preferred, for the
sake of expressiveness: \lstinline{bool x : 1}.
\end{enumerate}

\end{coderule}

\pagebreak
\section{Commenting Guidelines}

\pagebreak
\printbibliography

\pagebreak
\begin{appendix}
   \pagenumbering{gobble}
\end{appendix}

\end{document}