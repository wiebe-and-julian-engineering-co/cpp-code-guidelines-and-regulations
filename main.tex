\documentclass[12pt]{article}

\usepackage[style=ieee]{biblatex}
\addbibresource{main.bib}

\usepackage[T1]{fontenc}
\usepackage{lmodern}
\renewcommand*\familydefault{\sfdefault}

\usepackage{hyperref}
\hypersetup{
    colorlinks,
    citecolor=black,
    filecolor=black,
    linkcolor=black,
    urlcolor=black
}

\usepackage[margin=1in]{geometry}
\usepackage{fancyhdr}

\pagestyle{fancy}
\lhead{}
\chead{}
\rhead{}
\lfoot{\footnotesize Draft}
\cfoot{}
\rfoot{\thepage}
\renewcommand{\headrulewidth}{0pt}
\renewcommand{\footrulewidth}{0pt}

\newcommand{\cppuntil}[1]{\footnotesize\textit{(until C++#1)}\normalsize}
\newcommand{\cppsince}[1]{\footnotesize\textit{(since C++#1)}\normalsize}
\newcommand{\requirementpriority}[1]{\emph{Required-Level #1}}

\title{
      C++ Code Guidelines and Regulations \\
      \begin{large}
         Safety-Critical C++ for Embedded Systems
      \end{large}
}
\date{\today}
\author{Wiebe and Julian Engineering Cooperation}

\begin{document}
\pagenumbering{gobble} % No numbering on title page
\maketitle

\newpage
\pagenumbering{Roman} % Required to be roman until table of contents
\tableofcontents
\newpage
\pagenumbering{arabic} % Required to be arabic after table of contents

\newpage
\section{Introduction}
C++ is a general-purpose programming language. It has a wide vocabulary of
features, some of which are inspired by other language and some inspired other
languages. C++ is commonly called a superset of C, which in many respects
is true.

Because of the nature of being a superset of C, C++ has many
close-to-the-hardware programming capabilities. These low-level capabilities
must be handled with discipline and sanity. This document is used as a guide
and reference for C++ programmers in \textit{Wiebe and Julian Engineering Co.}
to achieve sane and safe C++ programs.

Many C++ programming aspects are evaluated in this document, included but not
limited by: readability, maintainability, portability, efficiency and security. \\
C++ code using the hwstl guidelines is also referred to as \textit{hwstl C++}.
Said C++ code is meant to be conform to all rules that are per definition
\requirementpriority{0} \\
The following topics will be covered in the C++ Code Guidelines and regulations
for Safety-Critical C++ for embedded sytems:
\begin{enumerate}
   \item Using the C++ standard in embedded systems,
   \item writing sane and safe C++ code, and
   \item following design goals for well written production code.
\end{enumerate}
\subsection{Requirement Levels}
Each formulation of a rule has an associated requirement priority, the
following priorities are formulated:
\begin{enumerate}
   \item \requirementpriority{0}: every rule with this priority is mandated to
be followed by all \textit{hwstl C++}.
   \item \requirementpriority{1}: all \textit{hwstl C++} made by \textit{Wiebe
and Julian Engineering Co.} is mandated to follow this rule. Exceptions can
only be made after inquiring a lead engineer.
   \item \requirementpriority{2}: it is advised to follow rules with this
priority level. Cases where it might be better to not follow the rule are not
required to be rapported.
\end{enumerate}

\subsection{Credility}
Thoughout this document, authors and books are referenced to support the
underlying guideline, in order to maintain credibility.
\newpage
\section{C++ Code Guidelines and Regulations}
\subsection{Reading Guide}
This section describes how the guides and regulations should be read and
interpreted. There are a couple of keywords that denote the urgency of
following a requirement. Such as:
\bigbreak
\textbf{Test:} Hello world
\subsubsection{C++ Versioning}
This guide takes modern C++ as of C++17 into account for coding practices. But
some platforms may not have any compilers with support for the latest C++
version available. Since some practices may be banned in C++17 while they might
be advised to use in C++14. These rules will be annotated with \cppuntil{14},
\cppsince{17} or both. This may be applied with any C++ version.

\newpage
\printbibliography

\newpage
\begin{appendix}
   \pagenumbering{gobble}
\end{appendix}

\end{document}